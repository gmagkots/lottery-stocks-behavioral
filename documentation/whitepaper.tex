\documentclass[12pt]{article}
\usepackage{amsmath}
\usepackage{txfonts}
\usepackage{color}
\usepackage{fancybox}
\usepackage{fullpage}
\usepackage{url}
\usepackage{setspace}
\usepackage{graphicx}
\usepackage[T1]{fontenc}
\usepackage{titlesec}
\usepackage{fancyhdr}
\usepackage{dsfont}
\usepackage{comment}
\usepackage{enumerate}
%\usepackage{slashbox}
%\DeclareMathSizes{12}{13}{10}{09}
%\everymath{\displaystyle}


% section format and equation numbering (within section)
\titleformat{\section}{\Large\bfseries}{}{1em}{}
\titleformat{\subsection}{\large}{}{0em}{}
\numberwithin{equation}{section}

% fullpage in inches
\addtolength{\topmargin}{-1.1cm}
\addtolength{\textheight}{2cm}


\begin{document}

\onehalfspacing
%\doublespacing

\begin{center}
\huge\textbf{Lottery stocks}
\end{center}

\section{Project idea}

\indent\par This project attempts to provide a potential explanation
to the value premium puzzle. The overall idea is to investigate the
behavior and strategies of two different types of investors/analysts
(agents hereafter). The first type includes agents who are thriving
among their peers. They are the top performers for given tasks, and
desire to maintain their status. Lets label this group ``top
performers''. The second type includes agents who are not top
performers, and desire to improve their status and performance among
their peers. Let this group be labeled as ``status-seekers''.

Our hypothesis is two-fold. We assume that the top performers adopt
strategies that are adequate enough to outperform their peers, and
help maintain their status at the same time. In other words, they
seek less risky choices (fearing a status loss in case of failure)
with potential high returns (to continue outperforming). On the
other hand, the status seekers are aware of the top performers'
strategy. Their skill level does not allow them to outperform their
peers by adopting a similar strategy. Thus, we assume that they are
forced to seek more risky choices to improve their status.

The next series of assumptions try to mock up a setup to help
explain the observed trends of the value premium puzzle. Choices of
higher risk could possibly have lower average returns compared to
top performer strategies, a feature assumed in order to account for
the top performers' dominance on average. However, it is possible
that the distribution of returns from high-risk choices
asymptotically dominates the corresponding low-risk distribution.
The high-risk pdf could either be more right-skewed or have thicker
tails compared to the low-risk pdf. As a result, a status seeker
being fully aware that cannot outperform his peers with low-risk
strategies, attempts high-risk strategies with a ``lottery''-type of
payoff. The reason is that he values status more than average
returns, and aims to win the lottery as the only alternative option
to outperform his peers.

We would like to test these hypotheses in the data. The problem
setup implies that top performers invest on value stocks, while
status seekers invest on growth stocks with potential lottery
payoffs. Currently, we plot these distributions and perform tests to
verify our claim.

\section{Description of data}

\indent\par We choose two dates (cross-section points) to calculate
stock returns. Growth and value stocks are identified through the
market to book ratio. We consider the ratio's value on the initial
date only to label stocks as growth or value (those that qualify,
not all stocks). There is the thought of using the M/B ratio on both
dates, but that would exclude firms whose stock was initially growth
and has evolved to value (or at least no longer growth) within the
period considered.

In order to calculate the M/B ratio, we merge data from the Daily
Stock File (DSF) and its header (DSFHDR) in CRSP with the
Fundamental Annual (FUNDA) data in COMPUSTAT\footnote{The linking
set CCMXPF\_LINKTABLE from CRSP is used to properly link the
libraries.}. Firms that have been liquidated in between the given
cross-section points are excluded, but care is taken to include
events such as M\&As that result in changes to data headers
(tickers, names, etc). Penny stocks (below a set threshold,
typically \$10.00) are excluded, but a limited number of stocks
above the threshold price and negative book value are included.
However, the ranking among stocks is done by the absolute value of
M/B. Further filters include
\begin{enumerate}
\item Choose only Ordinary Common Shares (share codes
``10'',``11'')\footnote{Other authors include preferred stock, other
tangible assets, etc.}
\item Include only certain link types (``LC'', ``LU'', ``LS'', ``LN'')
\item Exclude dummy IIDs (existing company, non-existent security)
\item Sanity check for price (allow negative for bid/offer average)
\item Sanity check for outstanding shares
\item Market cap upper and lower limits
\end{enumerate}

The M/B ratio calculation includes unavoidably numbers reported on
different dates. Since the accessible data on firm fundamentals are
only those with annual periodicity (due to USC subscription to
WRDS), we use the past fiscal-year-end reporting date that is
closest to our chosen cross section date. This approach coincides
most of the times with the standard method that Fama \& French have
established, when the reported fiscal-year-end date is close to
December. However, I make sure that I still get the closest past
fiscal-year-end date for cases where this date is end of March for
some firms.

Given the difference between the selected date (first cross-section
point) $CS$ and the most recent past fiscal-year-end date $FYE$, a
straightforward way to define the M/B ratio is
\begin{align*}
\text{M/B}^*&=\frac{\text{stock price}(CS)\times\text{outstanding
shares}(CS)}{|\text{book value per
share}(FYE)|\times\text{outstanding
shares}(FYE)}\\
&=\frac{\text{market value CRSP}(CS)}{\text{book value
COMPUSTAT}(FYE)}
\end{align*}
However, there are cases where a firm has more than one class of
shares, where one or more classes are not traded. Data for such
classes are not included in CRSP data, but are included in the
calculation of the firm's book value in COMPUSTAT. In order to
correct approximately for such cases, we use the following M/B ratio
\begin{align*}
\text{M/B}=\frac{\text{market value COMPUSTAT}(FYE)}{\text{market
value CRSP}(FYE)}\text{M/B}^*
\end{align*}







\end{document}
